% ------------------------------------------------------------------------
%												CONFIGURACI�N
% ------------------------------------------------------------------------
%
\documentclass[12pt,a4paper,spanish]{book}
%
%----------------------------------------------------------
% This is a sample document for the standard LaTeX Book Class
% Class options
%       --  Body text point size:
%                        10pt (default), 11pt, 12pt
%       --  Paper size:  letterpaper (8.5x11 inch, default)
%                        a4paper, a5paper, b5paper,
%                        legalpaper, executivepaper
%       --  Orientation (portrait is the default):
%                        landscape
%       --  Printside:   oneside, twoside (default)
%       --  Quality:     final(default), draft
%       --  Title page:  titlepage, notitlepage
%       --  Columns:     onecolumn (default), twocolumn
%       --  Start chapter on left:
%                        openright(no, default), openany
%       --  Equation numbering (equation numbers on right is the default):
%                        leqno
%       --  Displayed equations (centered is the default):
%                        fleqn (flush left)
%       --  Open bibliography style (closed bibliography is the default):
%                        openbib
% For instance the command
%          \documentclass[a4paper,12pt,reqno]{book}
% ensures that the paper size is a4, fonts are typeset at the size 12p
% and the equation numbers are on the right side.

\usepackage{amsmath,amssymb,amsfonts,mathrsfs}
\usepackage{graphicx}
\usepackage{anysize}
%\usepackage[utf8]{inputenc}
\usepackage[latin1]{inputenc}
\usepackage[spanish]{babel}  %Quita el comentario a esta l�nea y comenta la siguiente si quieres que los n�meros romanos aparezcan en may�sculas
%\usepackage[spanish, es-lcroman]{babel} 
\usepackage[bf]{caption}
\usepackage{subfigure}
\usepackage{rotating}
\usepackage{fontenc}
%\usepackage{mathaccent}
\usepackage{titlesec}
\usepackage{multirow} %Para poder centrar verticalmente el contenido de las celdas de una tabla
\usepackage{fancyhdr} %Para personalizar los encabezados y pies de p�gina
\usepackage{acronym}  %Para expandir autom�ticamente los acr�nimos
\usepackage[titletoc]{appendix} %Para que cambie el t�tulo y la forma de numerar los ap�ndices


%Define el formato de los encabezados y pies de p�gina del �ndice y la lista de acr�nimos
\fancypagestyle{plain}{ %Encabezado y pie para el �ndice y acr�nimos
 \fancyhf{}  %Elimina encabezdo y pie (menos la l�nea del encabezado)
 \renewcommand{\headrulewidth}{0pt} %Elimina la l�nea del encabezado
 \fancyfoot[LE,RO]{\thepage}
}

\marginsize{2.5cm}{2.5cm}{2.5cm}{2.5cm} %Margenes pagina

\linespread{1.5} %Espaciado entre lineas

% Esto es para redefinir los t�tulos que Latex pone por defecto
%\addto\captionsspanish{\renewcommand{\contentsname}{Contenido}}
\renewcommand{\tablename}{Tabla}

%Esto es para redefinir el formato en que se presenta el t�tulo de los cap�tulos
\titleformat{\chapter}[hang]{\Huge\bfseries}{\fontsize{24}{60} Cap�tulo \thechapter{. }}{0pt}{\fontsize{24}{60}}


\newcommand{\reff}[1]{Figura \ref{#1}}
\newcommand{\refe}[1]{(\ref{#1})}
\renewcommand{\captionfont}{\small}

%Si el idioma es espa�ol las listas aparecen con un cuadradito.
%En ingl�s aparecen con un bullet...Esto redefine el bullet a cuadrado.
\renewcommand{\labelitemi}{\tiny{$^\blacksquare$}}

%\renewcommand{\figurename}{Fig.}

\hyphenation{mo-du-la-tion pa-ra-me-te-ri-zed res-pon-se cha-rac-ter par-ti-cu-la-ri-zing de-ve-lo-ped a-ve-ra-ging pro-ba-bi-li-ty me-cha-nism par-ti-cu-lar know-led-ge pro-ducts inter-operable to-po-lo-gy cha-rac-te-ri-zed stra-te-gy rea-li-za-da si-mu-la-cio-nes pro-pues-tas pro-pues-to cen-tra-li-za-das igua-la-do-res su-mi-nis-tro va-lo-ra-do in-ter-fe-ren-cia par-ti-cu-la-ri-ties mul-ti-plexing ins-ti-tu-te ca-rri-er maxi-mum} %es un modo burdo de definir como romper una palabra

%\sloppy %permite que sea permisivo con las l�neas sueltas


%%Para que las vi�etas del segundo nivel aparezcan como a),b)...y no (a), (b)
\renewcommand{\theenumii}{\alph{enumii}}
\renewcommand{\labelenumii}{\textsf{\theenumii})}


%%----------------------------------------------------------
%\newtheorem{theorem}{Theorem}
%\newtheorem{acknowledgement}[theorem]{Acknowledgement}
%\newtheorem{algorithm}[theorem]{Algorithm}
%\newtheorem{axiom}[theorem]{Axiom}
%\newtheorem{case}[theorem]{Case}
%\newtheorem{claim}[theorem]{Claim}
%\newtheorem{conclusion}[theorem]{Conclusion}
%\newtheorem{condition}[theorem]{Condition}
%\newtheorem{conjecture}[theorem]{Conjecture}
%\newtheorem{corollary}[theorem]{Corollary}
%\newtheorem{criterion}[theorem]{Criterion}
%\newtheorem{definition}[theorem]{Definition}
%\newtheorem{example}[theorem]{Example}
%\newtheorem{exercise}[theorem]{Exercise}
%\newtheorem{lemma}[theorem]{Lemma}
%\newtheorem{notation}[theorem]{Notation}
%\newtheorem{problem}[theorem]{Problem}
%\newtheorem{proposition}[theorem]{Proposition}
%\newtheorem{remark}[theorem]{Remark}
%\newtheorem{solution}[theorem]{Solution}
%\newtheorem{summary}[theorem]{Summary}
%\newenvironment{proof}[1][Proof]{\textbf{#1.} }{\ \rule{0.5em}{0.5em}}
%%----------------------------------------------------------