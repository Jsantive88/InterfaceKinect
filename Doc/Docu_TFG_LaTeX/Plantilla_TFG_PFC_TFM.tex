%%%%%%%%%%%%% Configuraci�n %%%%%%%%%%%%%%%%%%%%%%
% ------------------------------------------------------------------------
%												CONFIGURACI�N
% ------------------------------------------------------------------------
%
\documentclass[12pt,a4paper,spanish]{book}
%
%----------------------------------------------------------
% This is a sample document for the standard LaTeX Book Class
% Class options
%       --  Body text point size:
%                        10pt (default), 11pt, 12pt
%       --  Paper size:  letterpaper (8.5x11 inch, default)
%                        a4paper, a5paper, b5paper,
%                        legalpaper, executivepaper
%       --  Orientation (portrait is the default):
%                        landscape
%       --  Printside:   oneside, twoside (default)
%       --  Quality:     final(default), draft
%       --  Title page:  titlepage, notitlepage
%       --  Columns:     onecolumn (default), twocolumn
%       --  Start chapter on left:
%                        openright(no, default), openany
%       --  Equation numbering (equation numbers on right is the default):
%                        leqno
%       --  Displayed equations (centered is the default):
%                        fleqn (flush left)
%       --  Open bibliography style (closed bibliography is the default):
%                        openbib
% For instance the command
%          \documentclass[a4paper,12pt,reqno]{book}
% ensures that the paper size is a4, fonts are typeset at the size 12p
% and the equation numbers are on the right side.

\usepackage{amsmath,amssymb,amsfonts,mathrsfs}
\usepackage{graphicx}
\usepackage{anysize}
%\usepackage[utf8]{inputenc}
\usepackage[latin1]{inputenc}
\usepackage[spanish]{babel}  %Quita el comentario a esta l�nea y comenta la siguiente si quieres que los n�meros romanos aparezcan en may�sculas
%\usepackage[spanish, es-lcroman]{babel} 
\usepackage[bf]{caption}
\usepackage{subfigure}
\usepackage{rotating}
\usepackage{fontenc}
%\usepackage{mathaccent}
\usepackage{titlesec}
\usepackage{multirow} %Para poder centrar verticalmente el contenido de las celdas de una tabla
\usepackage{fancyhdr} %Para personalizar los encabezados y pies de p�gina
\usepackage{acronym}  %Para expandir autom�ticamente los acr�nimos
\usepackage[titletoc]{appendix} %Para que cambie el t�tulo y la forma de numerar los ap�ndices


%Define el formato de los encabezados y pies de p�gina del �ndice y la lista de acr�nimos
\fancypagestyle{plain}{ %Encabezado y pie para el �ndice y acr�nimos
 \fancyhf{}  %Elimina encabezdo y pie (menos la l�nea del encabezado)
 \renewcommand{\headrulewidth}{0pt} %Elimina la l�nea del encabezado
 \fancyfoot[LE,RO]{\thepage}
}

\marginsize{2.5cm}{2.5cm}{2.5cm}{2.5cm} %Margenes pagina

\linespread{1.5} %Espaciado entre lineas

% Esto es para redefinir los t�tulos que Latex pone por defecto
%\addto\captionsspanish{\renewcommand{\contentsname}{Contenido}}
\renewcommand{\tablename}{Tabla}

%Esto es para redefinir el formato en que se presenta el t�tulo de los cap�tulos
\titleformat{\chapter}[hang]{\Huge\bfseries}{\fontsize{24}{60} Cap�tulo \thechapter{. }}{0pt}{\fontsize{24}{60}}


\newcommand{\reff}[1]{Figura \ref{#1}}
\newcommand{\refe}[1]{(\ref{#1})}
\renewcommand{\captionfont}{\small}

%Si el idioma es espa�ol las listas aparecen con un cuadradito.
%En ingl�s aparecen con un bullet...Esto redefine el bullet a cuadrado.
\renewcommand{\labelitemi}{\tiny{$^\blacksquare$}}

%\renewcommand{\figurename}{Fig.}

\hyphenation{mo-du-la-tion pa-ra-me-te-ri-zed res-pon-se cha-rac-ter par-ti-cu-la-ri-zing de-ve-lo-ped a-ve-ra-ging pro-ba-bi-li-ty me-cha-nism par-ti-cu-lar know-led-ge pro-ducts inter-operable to-po-lo-gy cha-rac-te-ri-zed stra-te-gy rea-li-za-da si-mu-la-cio-nes pro-pues-tas pro-pues-to cen-tra-li-za-das igua-la-do-res su-mi-nis-tro va-lo-ra-do in-ter-fe-ren-cia par-ti-cu-la-ri-ties mul-ti-plexing ins-ti-tu-te ca-rri-er maxi-mum} %es un modo burdo de definir como romper una palabra

%\sloppy %permite que sea permisivo con las l�neas sueltas


%%Para que las vi�etas del segundo nivel aparezcan como a),b)...y no (a), (b)
\renewcommand{\theenumii}{\alph{enumii}}
\renewcommand{\labelenumii}{\textsf{\theenumii})}


%%----------------------------------------------------------
%\newtheorem{theorem}{Theorem}
%\newtheorem{acknowledgement}[theorem]{Acknowledgement}
%\newtheorem{algorithm}[theorem]{Algorithm}
%\newtheorem{axiom}[theorem]{Axiom}
%\newtheorem{case}[theorem]{Case}
%\newtheorem{claim}[theorem]{Claim}
%\newtheorem{conclusion}[theorem]{Conclusion}
%\newtheorem{condition}[theorem]{Condition}
%\newtheorem{conjecture}[theorem]{Conjecture}
%\newtheorem{corollary}[theorem]{Corollary}
%\newtheorem{criterion}[theorem]{Criterion}
%\newtheorem{definition}[theorem]{Definition}
%\newtheorem{example}[theorem]{Example}
%\newtheorem{exercise}[theorem]{Exercise}
%\newtheorem{lemma}[theorem]{Lemma}
%\newtheorem{notation}[theorem]{Notation}
%\newtheorem{problem}[theorem]{Problem}
%\newtheorem{proposition}[theorem]{Proposition}
%\newtheorem{remark}[theorem]{Remark}
%\newtheorem{solution}[theorem]{Solution}
%\newtheorem{summary}[theorem]{Summary}
%\newenvironment{proof}[1][Proof]{\textbf{#1.} }{\ \rule{0.5em}{0.5em}}
%%---------------------------------------------------------- 
%El fichero config.tex contiene:
% 	Define los par�meros comunes a todo el documento (tipo de letra, tama�o, m�rgenes, etc.). Tambi�n
%   se incluyen los paquetes que el compilador necesita para manejar cosas "extra�as" como las figuras.


%%%%%%%%%%%%%%%%%%%%%%%%%%%%%%%%%%%%%%%%%%%%%%%%%%%%%%%%%%%%%%%%%%%%%%%%%%%%%%%%%%%%%%%%%%%%%%%%%%%%%%
%%%%%%%%%%%%%%%%%%%%%%%%%%%%%%%%%%%%%%%%% COMIENZA EL DOCUMENTO
%%%%%%%%%%%%%%%%%%%%%%%%%%%%%%%%%%%%%%%%%%%%%%%%%%%%%%%%%%%%%%%%%%%%%%%%%%%%%%%%%%%%%%%%%%%%%%%%%%%%%%
\begin{document}

\parindent 0cm  %elimina el indentado de p�rrafos
\pagestyle{empty}  %Para que no numere ni ponga encabezado ni los agradecimientos ni la dedicatoria


%%%%%%%%%%%%%%%%%%%%%%%%%%%%%%%%%%%%%%%%%%%%%
%% Titulo & Agradecimientos
%%%%%%%%%%%%%%%%%%%%%%%%%%%%%%%%%%%%%%%%%%%%%
\emph{ }

\vspace{10mm}
\textbf{\huge{TFG: Interfaz grafica Kinect-Persona}}
\vspace{2mm}

\cleardoublepage

Agradezco todo la ayuda, el apoyo y confianza de mi tutor, Sergio Escalera. Que a pesar de las diferentes dificultades mee ayudado y guiado en este proyecto que para mi ha sido uno de los grandes desconocidos. Permitiendome aprender y crecer en este campo.
Como la ayuda desde las cosas mas basica a otra no tan tribiales a Albert Clapes.

\newpage

  %Aqu� van los agradecimientos

%%%%%%%%%%%%%%%%%%%%%%%%%%%%%%%%%%%%%%%%%%%%%
%% Dedicatoria
%%%%%%%%%%%%%%%%%%%%%%%%%%%%%%%%%%%%%%%%%%%%%
\begin{flushright}
\textit{Me gustaria dedicar este proyecto tanto a mi familia.\\
Junto con todos los que me han apoyado y ayudado para sacarlo adelante.\\
Sin su ayuda no se si lo habria conseguido.\\}
\end{flushright}


\cleardoublepage
\pagenumbering{roman}
\pagestyle{plain} %define el formato de la cabecera
%\addcontentsline{toc}{chapter}{\hspace{5.26mm}Lista de Acr�nimos}
\renewcommand{\contentsname}{�ndice}  %Quitar el comentario a esta l�nea si se quiere que en lugar de "Contenidos" aparezca "�ndice"
\tableofcontents

\cleardoublepage

%Inclusi�n de la lista de acr�nimos
%Lista de acr�nimos 
\chapter*{Acr�nimos}

\begin{acronym}[DLMS/COSEMM]
	\acro{GEI}{Grau d'Enginyeria Inform�tica}
	\acro{UB}{Universidad de Barcerlona}
	\acro{PCL}{Point Cloud Librery}
\end{acronym}


\cleardoublepage
%Cambia el estilo del encabezado
\pagenumbering{arabic}
\pagestyle{fancy}
\setlength{\headheight}{15pt}
\fancyhf{}
\fancyfoot[LE,RO]{\thepage}
\fancyhead[LE,RO]{\nouppercase{\leftmark}}
\renewcommand{\headrulewidth}{0.1pt} %Elimina la l�nea del encabezado


%Aqu� se deben ir incluyendo los ficheros que contienen cada uno de los cap�tulos y las referencias y ap�ndices (si los hubiere)
% ------------------------------------------------------------------------
%                             Introducci�n
% ------------------------------------------------------------------------
\chapter{Introducci�n}  %NOMBRE DEL CAP�TULO
\section{Introducci�n} 



\subsection{Motivaciones}


%\section{Objetivos}

\newpage


%%%%%%%%%%%%%%%%%%%%%%%%%%%%%%%%%%%%%%%%%%%%%%%%%%%%%%%%
%%%Esto es para incluir una referencia bibliogr�fica
%%%%%%%%%%%%%%%%%%%%%%%%%%%%%%%%%%%%%%%%%%%%%%%%%%%%%%%%
%\cite{clave_de_la_referencia}

%%%%%%%%%%%%%%%%%%%%%%%%%%%%%%%%%%%%%%%%%%%%%%%%%%%%%%%%
%%%Esto es para incluir hacer referencia a una figura, tabla o ecuaci�n
%%%%%%%%%%%%%%%%%%%%%%%%%%%%%%%%%%%%%%%%%%%%%%%%%%%%%%%%
%\reff{etiqueta_de_la_figura} %Para figura
%\refe{etiqueta_de_la_ecuaci�n} %Para ecuaci�n
%\ref{etiqueta_de_la_figura} %Para tabla. En las dos anteriores se incluye autom�ticamente el texto Fig. (para el caso de la figura) y los par�ntesis para el caso de las ecuaciones. Aqu� hay que incluir a mano el texto Tabla. Ejemplo: como se indica en la Tabla \ref{etiqueta_de_la_tabla}


%%%%%%%%%%%%%%%%%%%%%%%%%%%%%%%%%%%%%%%%%%%%%%%%%%%%%%%%
%%%Esto es para incluir una figura
%%%%%%%%%%%%%%%%%%%%%%%%%%%%%%%%%%%%%%%%%%%%%%%%%%%%%%%%
%\begin{figure}[!htb]
%\centering
%\includegraphics[width=15cm]{Nombre_del_fichero.eps}
%\captionstyle{center} \caption{\small{Este es el pie de figura}}
%\label{figura_1}
%\end{figure}

%%%%%%%%%%%%%%%%%%%%%%%%%%%%%%%%%%%%%%%%%%%%%%%%%%%%%%%%
%%%Esto es para incluir una tabla
%%%%%%%%%%%%%%%%%%%%%%%%%%%%%%%%%%%%%%%%%%%%%%%%%%%%%%%%

%\begin{center}
    %\begin{tabular}{ | l | l | l | l|}
    %\hline
    %Elemento 1,1 & Elemento 1,2 & Elemento 1,3 & Elemento 1,4 \\ \hline
    %Elemento 2,1 &  Elemento 2,2 &  Elemento 2,3 &  Elemento 2,4 \\ \hline
    %Elemento 3,1 &  Elemento 3,2 &  Elemento 3,3 &  Elemento 3,4 \\ \hline
    %\end{tabular}
%\end{center}
%%%%%%%%%%%%%%%%%%%%%%%%%%%%%%%%%%%%%%%%%%%%%%%%%%%%%%%%
%%%Esto es para incluir dos figuras en paralelo
%%%%%%%%%%%%%%%%%%%%%%%%%%%%%%%%%%%%%%%%%%%%%%%%%%%%%%%%
%\begin{figure}[!h]
%\centering \subfigure[]{\includegraphics[width=8cm]{nombre_fichero_figura1.eps}}
%\subfigure[]{\includegraphics[width=8cm]{nombre_fichero_figura2.eps}}
%\captionstyle{center} \caption{\small{Este es el pie de figura}} \label{figura_1}
%\end{figure}
%

%%%%%%%%%%%%%%%%%%%%%%%%%%%%%%%%%%%%%%%%%%%%%%%%%%%%%%%%
%%%Esto es para incluir una expresi�n matem�tica
%%%%%%%%%%%%%%%%%%%%%%%%%%%%%%%%%%%%%%%%%%%%%%%%%%%%%%%%
%\begin{equation}
%x=\sum_{i=0}^{9}i^2,
%    \label{ecuacion_1}
%\end{equation}


% ------------------------------------------------------------------------
%                             Objetivos
% ------------------------------------------------------------------------
\chapter{Objetivos}  %NOMBRE DEL CAP�TULO
\section{Objetivos} 


\newpage
% ------------------------------------------------------------------------
%                             Desarrollo
% ------------------------------------------------------------------------
\chapter{Desarrollo}  %NOMBRE DEL CAP�TULO
\section{Desarrollo} 


\newpage
% ------------------------------------------------------------------------
%                             PruebasResultados
% ------------------------------------------------------------------------
\chapter{PruebasResultados}  %NOMBRE DEL CAP�TULO
\section{PruebasResultados} 


\newpage
% ------------------------------------------------------------------------
%                             Conclusiones
% ------------------------------------------------------------------------
\chapter{Conclusiones}  %NOMBRE DEL CAP�TULO
\section{Conclusiones} 

\newpage


%Cambia el T�tulo Cap�tulo por Ap�ndice (no lo har� autom�ticamente con \appendix porque se forzo la definici�n de Cap�tulo en config.tex)
\titleformat{\chapter}[hang]{\Huge\bfseries}{\fontsize{24}{60} Ap�ndice \thechapter{. }}{0pt}{\fontsize{24}{60}}
\appendix

% ------------------------------------------------------------------------
%                            Ap�ndice
% ------------------------------------------------------------------------
\chapter{Ap�ndice}
\label{Tapendice}


\section{Primera secci�n del ap�ndice}

%% ------------------------------------------------------------------------
%											REFERENCIAS
% ------------------------------------------------------------------------


\cleardoublepage

\addcontentsline{toc}{chapter} {Referencias}
\label{Referencias}


\renewcommand\bibname{Referencias}
\bibliography{Lista_de_referencias} %ENTRE LLAVES DEBE IR EL NOMBRE DE LA BASE DE DATOS QUE CONTIENE LAS REFERENCIAS
\bibliographystyle{IEEEtran}  %con numeros
%\bibliographystyle{IEEEpes}  %con numeros
%\bibliographystyle{amsalpha}  %con abreviaturas
%\bibliographystyle{apalike} %con el nombre entero autor + fecha, problema orden nombres

\end{document}
